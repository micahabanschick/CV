\documentclass[11pt]{article}
\usepackage{graphicx} % Required for inserting images
\setlength{\parindent}{0pt}
\usepackage{hyperref}
\usepackage{enumitem}
\usepackage[utf8]{inputenc} 
\usepackage[T1]{fontenc}
\usepackage[brazil]{babel}
\usepackage{lipsum}
\usepackage[left=1.0cm,top=1.0cm,right=1.0cm,bottom=1.0cm]{geometry}




%by: Aline R. Antunes

\begin{document}
\begin{center}
    \textbf{Micah Banschick}\\ 
    \hrulefill
\end{center}

\begin{center}
    Stamford, CT 06905 \textbullet \ \href{mailto:micah.banschick@uconn.edu}{micah.banschick@uconn.edu} \textbullet \ (914) 714-3677 \textbullet \ \href{https://www.linkedin.com/in/micah-banschick/}{linkedin.com/in/micah-banschick/}
\end{center}

\vspace{0.5pt}

\begin{center}
    \textbf{Education}
\end{center}
\textbf{University of Connecticut} \hfill Storrs, CT

B .Sc. in Physics and Mathematics (Honors). \hfill Graduation Date: May 2027

Thesis: [To be determined] - \textit{Advisor: Prof. Peter Schweitzer}

\vspace{12pt}


\begin{center}
    \textbf{Research}
\end{center}
\textbf{Optical Signatures of Ion-neutral Collisions} - \textit{Advisor: Dr. Benjamin Prince} \hfill Kirtland Air Force Base, NM

\textit{Air Force Research Laboratory} \hfill May 2024 – August 2024
\begin{itemize}[noitemsep, topsep=0pt, partopsep=0pt, parsep=0pt]
    \item Experimentally gathered optical emissions of $N^+_2 + N_2$ collisions for $N^+_2(B-X)$ and $N^+_2(A-X)$ between $16\ \frac{eV}{q}$ and $600\ \frac{eV}{q}$ kinetic energies using a $13.86\ MHz$ radio frequency ion source.
    \item Analyzed vibrational and rotational transitions through the generation of basis functions calculated in Diatomic, with subsequent simulation of the data accomplished with IgorPro.
\end{itemize}

\vspace{8pt}

\textbf{Merging Supermassive Black Hole Binary Systems} - \textit{Advisor: Prof. Jonathan Trump} \hfill Storrs, CT

\textit{University of Connecticut} \hfill January 2024 – Present
\begin{itemize}[noitemsep, topsep=0pt, partopsep=0pt, parsep=0pt]
    \item Calculated the spectral emission distributions of supermassive black holes by integrating radially through their mini-disks.
    \item Modeled black hole luminosity fluctuations using Python packages such as matplotlib and astropy.
\end{itemize}

\vspace{8pt}

\textbf{Quantum Computation Power with Qubit} - \textit{Advisor: Prof. Lea Ferreira dos Santos} \hfill Storrs, CT

\textit{Yale University} \hfill January 2024 – Present
\begin{itemize}[noitemsep, topsep=0pt, partopsep=0pt, parsep=0pt]
    \item Analyzed the matrix symmetry of the Kerr Hamiltonian representing the observables of a qubit.
    \item Demonstrated code redundancies using contour plots in Python and Mathematica.
\end{itemize}

\vspace{8pt}

\textbf{Predicting Stock Trends Using LSTM-Neural Networks} - \textit{Advisor: Prof. Phillip Bradford} \hfill Stamford, CT

\textit{University of Connecticut} \hfill May 2022 – July 2022
\begin{itemize}[noitemsep, topsep=0pt, partopsep=0pt, parsep=0pt]
    \item Used TensorFlow and LSTM-Neural Networks to predict Stock trends.
    \item Collaborated weekly with the research supervisor about how to improve the program.
\end{itemize}

\vspace{8pt}

\textbf{Psychology of Human Expression} - \textit{Advisor: Yuvalal Liron} \hfill Rehovot, Israel

\textit{Weizmann Institute of Science} \hfill September 2016 - January 2017
\begin{itemize}[noitemsep, topsep=0pt, partopsep=0pt, parsep=0pt]
    \item Simulated over 500 instances of expression through mock-illustrations and analyzed results in MATLAB.
    \item Brainstormed, prepared, and presented discoveries and ideas to the research team weekly.
\end{itemize}

\vspace{12pt}

\begin{center}
    \textbf{Posters \& Presentations}
\end{center}

\textbf{Air Force Research Laboratory} - \textit{Advisor: Dr. Benjamin Prince} \hfill Poster Presentation, July 2024

\vspace{12pt}

\begin{center}
    \textbf{Awards \& Honors}
\end{center}

\textbf{Philips Scholar:} \$11,838.40 \hfill USRA, 2024

\textbf{Honors Scholar in Physics:} Thesis: [To be determined] \hfill University of Connecticut, 2023

\textbf{Annual Physics Award} \hfill University of Connecticut, 2023

\textbf{Annual Mathematics Award} \hfill University of Connecticut, 2023

\vspace{12pt}

\begin{center}
    \textbf{Skills \& Interests}
\end{center}

\textbf{Research:} Quantum Dynamics, Particle Physics, Supermassive Black Holes, QFT, Quantum Computing

\textbf{Technical:} Python, Mathematica, Matlab, LaTeX, IgorPro, Diatomic, Javascript, Java, Ruby, Excel Solver

\textbf{Laboratory:} Molecular Spectroscopy, Ion Beam Imaging, Radio Frequency Ion Sourcing, Wein Velocity Filtration
\pagenumbering{gobble}




\end{document}
